\section{Vorlesung 08.06.2017}

\subsection{Teil 1: Fitnesslandschaften}

\textbf{evol. Theorie:} Wachstumsrate in Räuber-Beute-Modellen + Reproduktion mit Variation + ''Survival of the fittests" $\rightarrow$ Wachstumsrate einer Population in einer gege. Umgebung
\\\\
\underline{Günther Wagner: Messtheorie von Fitness (measurement theorie)}
\\

X… Suchraum (genotyp|phänotyp), allgemein irgendeine Repräsentation der betrachteten Taxa
\\
Ähnlichkeitsstruktur $\sigma$\\
Fitnessfunktion: $f:x \rightarrow R$\\
mit R=totale geordnete Menge ($f_1, f_2 \in R: f_1 < f_2, f_1 > f_2, f_1 = f_2$)
\\

% pic 1

Begründer: Sewall Wright ($\sim$ 1930)\\
siehe zurückliegendes Bild: Individuum hat höhere Wahrscheinlichkeit Erbgut in nächste Generation zu übertragen (''Verbesserung")\\

\subsubsection{Genetische Algorithmen}
Idee: benutze künstliche Evolution um Optimierungsprobleme zu lösen

\begin{enumerate}
	\item Population $A \subseteq X$
	\item Nachfolgerpopulation von Kandidaten C(A)
	\item Selektiere die Besten bezüglich Fitnessfunktion $x \in C(A)$
	\item zurück zu 1.
\end{enumerate}

genetische Algorithmen, evolutionäre Progammierung (Rechenberg, Schwefel $\sim$ 1960/70)\\

\underline{geg.:} RNA oder Proteinsequenz $\alpha$
\underline{ges.:} Alle möglichen Strukturen x, die $\alpha$ einnehmen kann $\rightarrow$ Menge x von Konfigurationen\\
Energiefunktion $f: X \rightarrow R$\\
z.B. Loop basiertes Energiemodell für RNA Sekundärstrukturen\\

% pic 2

\underline{Lenskis E.Coli Zucht\footnote{\url{https://en.wikipedia.org/wiki/E._coli_long-term_evolution_experiment}}}\\
X… Menge von Gen oder Genomsequenzen\\
$\sigma$… Mutationen (hauptsächlich Substitution, Insertion, Deletion)\\
(X,$\sigma$)… Suchraum $\Leftrightarrow$ Graphen über \{A,G,T,C\}$^n$\\
mit n=Sequenzlänge\\
mit Kanten=Hammingdistanz 1 ($|$ Levensteindistanz 1)

\subsubsection{3D-Strukturen}
Proteinstruktur = ($\overrightarrow{x_1}, \overrightarrow{x_2}, ... , \overrightarrow{x_n}$) mit $\overrightarrow{x_1}$=3D-Koordinaten für Atom 1\\
\\
Constraint: Bindungswinkel, Bindungslängen\\
X… alle möglichen 3D-Einbettungen des Proteins\\
Nachbar: $||\overrightarrow{x} - \overrightarrow{x}'|| = \sum_{i} |\overrightarrow{x_i} - \overrightarrow{x_i}' | < \epsilon$ für gegebenes $\epsilon > 0$\\\\

Wenn RNA Sekundärstrukturen?\\
X… Menge aller erlaubten Strukturen [(,),.]\\
x$\sim$y wenn x und y sich durch ein Basenpaar unterscheiden $\Rightarrow$ Graph\\

\underline{Beispiel:}\\\\
% pic 3

\subsubsection{Optimierung auf Landschaften}
$\rightarrow$ max, min finden\\
%pic 4

\underline{Wie misst man Rauheit?}\\
Minimum: $\hat{x} \in X$ sodass $\forall y$ Nachbar von $\hat{x}: f\hat{(x)} \leq f(y)$\\
für metrischen (kontinuierlichen) Raum: $\forall y: |\hat{x} - y| < \epsilon$\\
Maximum: $f\hat{(x)} \geq f(y)$\\

\textbf{Was möchte man messen?}
\begin{itemize}
	\item \# lokale Minima, nur gut bei kleinen Instanzen, daher sampeln! (zufällige x wählen und bestimmen ab Minimum)
	\item mittlere Länge von \underline{adaptiven walks}\\
	$x_0,x_1, x_2, …, x_l$ sodass $x_i$ Nachbar von $x_{i-1}$ ist und $\underbrace{f(x_i) > f(x_{i-1})}_{*}$ i=1…l
	* für Fitness (für Energie $<$)
	\item Alternative: gradient walks (Weg des stelsten Anstiegs), Distanz zum ''nächstgelegenen" lokalen Min/Max
\end{itemize}

\subsubsection{Autokorrelationsfunktionen}
$x_0,x_1, x_2, …, x_l$ sodass $x_i$ Nachbar von $x_{i-1}$\\
betrachte Folge der Funktionswerte $f(x_0), f(x_1), …$
betrachte das als Signal (Zeitserie)\\
$\varrho(\tau)=\displaystyle \frac{<f(x_t) \cdot f(x_{t+\tau})>_t - <f>^2_t}{<f^2>_t - <f>^2_t}$\\
$<f_t>$ … Mittelwert über die $f(x)$\\
$<f_t> := \lim\limits_{T \rightarrow \infty} \frac{1}{T} \sum_{t=0}^{T-1} f(x_t)$\\
$<f(x_t) \cdot f(x_{t+\tau})> := \lim\limits_{T \rightarrow \infty} \frac{1}{T} \sum_{t=0}^{T-1} f(x_t)f(x_{t-\tau})$\\\\
X… Menge von Gen oder Genomsequenzen\\
$\sigma$… Graph, regulär (jedes x hat gleich viele Nachbarn D)\\
A… Adjazenz von (X,$\sigma$)\\

$\varrho(\tau)=\displaystyle \frac{(f(\frac{1}{D} \cdot A)^{\tau} \cdot f) - (f)^2}{(f^2) - (f)^2}$\\
$\Rightarrow$ leichter so auf Graphen als direkt auf Fitness (Funktion)\\

\underline{Korellationslänge:}\\
%pic 5

Funktion in Abhängigkeit der Verschiebung von $\tau$\\
$L_c=\displaystyle \sum_{\tau=0}^{\infty} \varrho(\tau)$\\
$(f) = \frac{1}{|X|} \sum_{x\in X} f(x)$\\
$(f^2) = \sum_{x \in X} f(x)^2$\\
$(f, \frac{1}{D}\cdot A \cdot f) := \sum_{x \in X} \sum_{y \in X} \frac{1}{D} f(y) \cdot A_{yx} \cdot f(x)$\\
wenn $(f)=0$ folgt vereinfachte Gleichung\\
$(f^2)=1 \rightarrow \varrho(\tau)= <f(x_t), f(x_{x_t - \tau})> = (f, \underbrace{\frac{1}{D} \cdot A \cdot f}_{\text{Graphstruktur}})$ 

\underline{Beispiel:}\\
%pic 6

Kostenfunktion in Fall 1 ändert sich stärker als in Falls 2\\
$\Rightarrow$ $L_R \simeq 2 L_T$\\\\

?(Länge Korrelationslängen $\rightarrow$ Lange Wege zum nächsten Minimum $\rightarrow$ gut!)

\subsection{Übung farbliche Ausprägung Katzenfell und beteiligte Gene}