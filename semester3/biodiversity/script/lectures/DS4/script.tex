\section{DS4}
\subsection{Artenreichtum messen}
\textbf{Was wir selten(st) schaffen}
\begin{itemize}
	\item Wir erfassen fast nie alle Organismen des untersuchten Systems (schon gar nicht einer Region)
	\item Wir erfassen so gut wie nie alle Taxa eines Systems (oft nur ausgewählte Gruppen)
	\item Die Erfassungsmethode richtet sich selten nach den Taxa / Arten, die am schwierigsten zu erfassen sind
\end{itemize}

\textbf{Welche Taxa / Arten sind schwierig zu erfassen?}
\begin{itemize}
	\item Generell: Seltene Arten
	\item Arten in schwer zugänglichen Bereichen des Ökosystems (tiefe Bodenschichten, Kronendach, ...)
	\item Arten in schwer bestimmbaren Zwischenstadien (Sporen, Samen, Nymphen)
	\item Arten, die sich auch als Adulte nur schwer bestimmen lassen
	\item Arten mit zeitlich sehr variabler Präsenz
	\item Arten mit räumlich sehr heterogener Präsenz (stark aggregiert an bestimmten Mikrostandorten)
\end{itemize}

\textbf{Problem: Was ist ein Individuum?}
\begin{itemize}
	\item Bei unitaren Organismen eindeutig: Form deterministisch
	\item Bei modularen Organismen nicht
	\begin{itemize}
		\item Bäume, Korallen, Schwämme
		\item Oft verzweigte, sich selbst wiederholende Strukturen
		\item Entwicklungsprogramm nicht vorhersagbar $\rightarrow$ indeterminiertes Wachstum
	\end{itemize}
	\item Modulare Organismen sind sehr häufig (Wälder, Grünländer, Korallenriffe, Moore)
\end{itemize}

\newpage
\textbf{Genet vs. Modul}
\begin{itemize}
	\item Genet:
	\begin{itemize}
		\item Genetisches Individuum; Produkt einer Zygote
		\item kann aus vielen Modulen bestehen (Polykormon)
		\item Beispiel: Nähnadel Gottes
	\end{itemize}
	\item Modul, z. B. bei Pflanzen:
	\begin{itemize}
		\item vegetatives Modul: Blatt, Knospe (in Blattachsel) und Internodium (fundamentales Modul = Phytomer)
		\item Generatives Modul: Blüte
		\item Äste: „kleine Bäumchen, die in einem großen Baum wurzeln
	\end{itemize}
	\item Ramet: Module, die sich vom Genet getrennt haben und m.o.w. unabhängig geworden sind
\end{itemize}

\subsection{Individuen/Module zählen}
\begin{itemize}
	\item Sessile Organismen:
	\begin{itemize}
		\item Plot abstecken und zählen
		\item Transekte
		\item „plotless sampling methods“
		\item Luftbilder (v.a. Bäume)
		\item Fernerkundung (v.a. Bäume oder Vegetationsstrukturen)
	\end{itemize}
	\item Bewegliche Organismen
	\begin{itemize}
		\item Diverse Fallen
		\item Fang-Wiederfang Methoden
		\item Sichtzählung / Transekte
		\item Akustische Kartierung
		\item Luftbilder
		\item Jagd- und Fangstatistiken
	\end{itemize}
\end{itemize}

\newpage
\textbf{Transektmethoden}
\begin{itemize}
	\item Schnitt-Transekte
	\begin{itemize}
		\item Sehr günstig bei Polykormonen
		\item Liefert: Anzahl, mittlere Kormongröße, Deckung
	\end{itemize}
	\item Lineare Transektplots
	\begin{itemize}
		\item Günstig bei kleinen „punktförmigen“ – z. B. Grasrameten
		\item Liefert: Anzahl
	\end{itemize}
\end{itemize}

\textbf{over pin frame}
\begin{itemize}
	\item v.a. im Grünland
	\item Gezählt werden die Berührungen von Organen (Blättern, Stengel, Blüte)
	\item Trennung nach Individuen nicht möglich
	\item Höhe der Berührung gibt auch Auskunft über vertikale Struktur
	\item Dauert lange, ist aber objektiver als Deckungsschätzungen
\end{itemize}

\textbf{Plot-less}
\begin{itemize}
	\item Viele plot-less Methoden sind ein Mischung aus einer Zufallsauswahl und Distanzmessungen, z. B.
	\begin{enumerate}
		\item Auwahl zufälliger Organismen $\rightarrow$ Messung der Distanz zum nächsten Nachbarn
		\item Auswahl zufälliger Orte $\rightarrow$ Messung der Distanz zu nächsten Organismus (s. Abbildung)
	\end{enumerate}
	\item Probleme:
	\begin{enumerate}
		\item Auswahl von zufälligen Individuen ist sehr schwierig
		\item Methode 2 wird sehr stark von isolierten Individuen beeinflusst.
		\item Lösung: z. B. T-Sampling (siehe Vorlesung)
	\end{enumerate}
\end{itemize}

\textbf{Point-Quarter}
\begin{itemize}
	\item Zufallspunkte i als Zentrum eines Kreuzes. Ingesamt n Zufallspunkte.
	\item Jeweils Distanz d zum nächsten Nachbar in Quadrant j messen.
	\item Vorteil: Man braucht weniger Zufallspunkte. Sehr effizient.
	\item Nachteil: Empfindlich gegenüber Abweichungen von der Zufallsverteilung.
\end{itemize}

\textbf{weitere Verfahren:}
\begin{itemize}
	\item Imaging: z. B. Laser Scanner, Spektralkamera, RGB Kamera, Thermokamera
	\item Multispektralaufnahme vom Flugzeug
	\item Akustisches Monitoring
\end{itemize}

\textbf{Ideale für Biodiversitätssampling}
\begin{itemize}
	\item Verschiedene Skalen für verschieden große Organismen
	\item Plots sind so klein, dass alle darin vorhandenen Arten und Individuen erfasst werden können
	\item Plots sind so zahlreich, dass alle vorkommenden Arten erfasst werden.
	\item Grundannahme: Alle Arten sind gleich gut detektierbar
\end{itemize}

\textbf{Whittaker-Plot}
\begin{itemize}
	\item Tastet Artenreichtum über verschiedene Skalen hinweg ab
	\item Abwandlungen:
	\begin{itemize}
		\item nested quadrat
		\item Long-Thin Plot
		\item modified whittaker plot
		\item ncvs Protokoll (siehe Vorlesung)
	\end{itemize}
\end{itemize}

\subsection{Maße für Artendiversität}
\begin{itemize}
	\item \textbf{Artenreichtum} (species richness)
	\begin{itemize}
		\item „richness“
		\item Chaos Schätzer
	\end{itemize}
	\item \textbf{Artendiversität} (species diversity)
	\begin{itemize}
		\item Shannon-Wiener Diversität
		\item Simpson Index
	\end{itemize}
	\item \textbf{Arten-Gleichverteilung} (eveness)
\end{itemize}

\newpage
\textbf{Arten-Akkumulationskurven}
\begin{itemize}
	\item Individuum-based
	\begin{itemize}
		\item Ein Individuum nach dem anderen sammeln
		\item Wenn ein neues dabei ist, Zähler eins höher setzen.
		\item Irgendwann wird man kaum noch neue Arten finden
	\end{itemize}
	\item Sample-based
	\begin{itemize}
		\item Ein Probe (mit potentiell mehreren Individuen) nach der anderen sammeln (oft Probeflächen)
		\item etc. siehe oben
	\end{itemize}
\end{itemize}

\textbf{\textcolor{red}{Fragen (siehe Vorlesung)}}
\begin{itemize}
	\item Ein Altwald und ein nachgewachsener Wald im Vergleich Warum sind die Kurven glatt? (rarefaction curve)
	\item Warum besteht einmal der Unterschied (individual-based) und einmal nicht (sample-based)?
	\item Warum ist die individual-based Kurve der Altwälder kürzer?
\end{itemize}