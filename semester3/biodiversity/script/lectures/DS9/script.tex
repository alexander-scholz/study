\section{DS9}
\subsubsection{Mark Westoby‘s LHS-System}
\textbf{Konzept:}
\begin{itemize}
	\item Drei „Leader“-Merkmale bilden drei voneinander unabhängige Achsen der Variabilität.
	\item Diese Achsen repräsentieren sogenannte „tradeoffs“ (d. h. unausweichliche Kompromisse)
\end{itemize}

\textbf{Die drei „tradeoffs“}
Eine Pflanze kann nicht...
\begin{itemize}
	\item ...dauerhafte Blätter haben, die gleichzeitig billig sind ($\rightarrow$ Spezifische Blattfläche – \textcolor{blue}{\textbf{L = leaf}})
	\item ...Blätter hoch über dem Boden positionieren, ohne vorher einen stabilen Stamm zu wachsen ($\rightarrow$ maximale Höhe – \textcolor{blue}{\textbf{H = height}})
	\item ...schwere Samen (mit hohem Startkapital) produzieren und gleichzeitig davon auch noch viele pro eingesetztem C-Kapital ($\rightarrow$ Samengewicht – \textcolor{blue}{\textbf{S = Seed}})
	\item Die drei Merkmale (L + H + S) sind mit diversen anderen eng korreliert (sie repräsentieren jeweils einen ganzen Syndromgradienten)
	\item Sie sind untereinander aber nicht korreliert (na, ja, siehe später)
	\item Vorteile des LHS Systems
	\begin{itemize}
		\item Quantitativ und merkmalsbasiert
		\item Tradeoffs sind real
		\item Die drei Merkmale sind diejenigen, für die die meisten Daten vorhanden sind.
	\end{itemize}
\end{itemize}

\textbf{Potentiell zusätzliche Achsen}
\begin{itemize}
	\item Wurzeln: Wurzeltiefe, Wurzelarchitektur, Feinwurzelmerkmale
	\item Blatt- und Zweigverhältnisse: Größen, Architektur, Zweiginvestition pro Blattfläche
	\item Hydraulik: Leitfähigkeit, Embolieresistenz, Anatomie
	\item Störungsresistenzen: thermisch: Feuer, mechanisch: Wind, Schneebruch, biotisch: Herbivorie, etc.)
	\item Kommunikation: Allelopathie ...usw.
\end{itemize}

\newpage
\textbf{Reproduktive Allokation (S)}
\begin{itemize}
	\item Arten mit kleinen Samen produzieren mehr davon (bei gleicher Allokation in Samen)
	\item Arten mit großen Samen sind als Sämlinge sehr konkurrenzkräftig
\end{itemize}

\textbf{Spezifische Blattfläche SLA (specific leaf area)}\\
$SLA=\frac{Blattflaeche}{Blatttrockengewicht}=\frac{1}{LMA}$ (\textcolor{red}{LMA=leaf mass avarage?})\\\\

Nachteile von hohem SLA
\begin{itemize}
	\item Geringerer Schutz gegen mechanische Belastung (Wind, Eiskristalle, Sand, Tritt, etc.)
	\item Hält nur geringe Wasserspannungen aus $\rightarrow$ empfindlich gegen Austrocknung
	\item Wenig verholzt $\rightarrow$ hohe Palatabilität (Nährwert für Herbivore hoch)
	\item Korreliert mit Kurzlebigkeit
\end{itemize}

\textbf{LMA und Relative Wachstumsrate}\\
Definition: Änderung der Biomasse über einen meist kleinen Zeitschritt im Verhältnis zu bestehenden Biomasse\\
$RGR=\frac{dW}{dt}\frac{1}{W}$\\
Normierung bez. bestehende Biomasse günstig für Vergleichbarkeit zwischen Organismen
\\\\
\textbf{SLA vs. mechanische Belastbarkeit („leaf toughness“)}
\begin{itemize}
	\item Relevant für z. B.
	\begin{itemize}
		\item Herbivorie
		\item Abrasionsfestigkeit (Sand, Eiskristalle)
		\item Trittfestigkeit
	\end{itemize}
	\item Messen als Kraft, die benötigt wird zum
	\begin{itemize}
		\item Auseinanderreißen (Zugfestigkeit)
		\item Punktieren (Punktationsfestigkeit)
		\item  Durchschneiden (Schneidfestigkeit)
	\end{itemize}
\end{itemize}

\newpage
\textbf{Mögliche Bestimmungsfaktoren}
\begin{itemize}
	\item Blattdicke
	\item Dichte und Anordnung der Blattnervatur (Blattaderung)
	\item Verstärkung der Blattnerven durch Sklerenchymfasern (lignifizierte Holzfasern, die die Gefäße umgeben)
	\item Einlagerung von Silikaten (v.a. bei Gräsern)
\end{itemize}

\textbf{Wovon es nicht abhing...}
\begin{itemize}
	\item Spezifische Blattfläche
	\item Blattdicke
	\item Kutikuladicke (oder Dicke von Parenchymen)
	\item Wassergehalt
	\item Blattdichte
\end{itemize}

\textcolor{blue}{\textbf{Blattnervatur war der entscheidende Faktor!}}