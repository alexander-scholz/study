\section{DS5}
\textbf{Chao‘s Schätzer}
\begin{itemize}
	\item Schätzt die „wahre“ Artenvielfalt
	\item Gleiche Datenstruktur wie vorher
	\item Die Gesamtartenzahl in der Probe wird extrapoliert
\end{itemize}

\textbf{Überlegung zur Artenvielfalt:} Welche Gemeinschaft ist diverser? Warum?
\\\\
\textbf{Shannon-Wiener Diversität}
\begin{itemize}
	\item Artenzahl S (“species richness”): Gesamtartenzahl, pro Fläche
	\item Shannon-Wiener-Diversität (H oder D Shannon ): Diversität abh. von Artenzahl und deren Häufigkeit (Rechenbeispiel siehe Vorlesung)
\end{itemize}

\textbf{Gini-Simpson's Diversitäts-Index D:} beschreibt die Wahrscheinlichkeit, mit der zwei zufällig ausgewählte Organismen der gleichen Art angehören (= Varianzmaß)\\
\textcolor{red}{Zwischen welchen Werten schwankt der Simpson-Index?}

\textbf{Was das Maß können sollte}
\begin{enumerate}
	\item Bei konstanter Artenzahl, Abundanz und Gleichverteilung (Eveness) aber variabler Anteile einzelner Arten, soll das Maß auch konstant bleiben
	\item Wenn die Gesamtabundanz abnimmt, wird das Maß kleiner
	\item Wenn nur die Gleichverteilung abnimmt (Abundanz, Artenzahl konstant), wird das Maß kleiner
	\item Wenn nur die Artenzahl abnimmt (Abundanz, Gleichverteilung konstant), wird das Maß kleiner
	\item Der Erwartungswert des Maßes sollte unabhängig von der Probenmenge sein
	\item Der Erwartungswert sollte einfach und präzise quantifizierbar sein
\end{enumerate}

\textcolor{blue}{Fazit bezüglich Shannon-Wiener und Simpson: Befriedigen alle Kriterien bis auf F2, d.h. wenn alle Arten (Artenzahl und Eveness konstant) in der Abundanz abnehmen, bleiben beide Indizes konstant!}

\newpage
\textbf{Eveness (= Gleichverteilung)}\\
Eveness ist ein Maß dafür, wie sich die vorkommenden Arten in ihren Abundanzen unterscheiden
\\\\
\textbf{(Shannon-) Eveness}\\
Wird typischerweise indirekt ausgerechnet:
\begin{itemize}
	\item Diversitätsmaß ist eine Mischung ist aus Artenzahl und Gleichverteilung (= Eveness)
	\item Wenn man die Artenzahl heraus rechnet, isoliert man die Eveness
\end{itemize}

2 Schritte:
\begin{enumerate}
	\item Shannon-Diversität für maximale Gleichverteilung: $D_{Shannon\ max}=ln(S)$
	\item Verhältnis der gemessenen zur maximalen Diversität = Gleichverteilung: $E=\frac{D_{Shannon}}{D_{Shannon\ max}}$
\end{enumerate}

\subsubsection{$\alpha$ $\beta$ $\gamma$ Diversität}
\begin{itemize}
	\item $\alpha$-Diversität: Diversität innerhalb der einzelnen Untersuchungseinheit (z. B. Plot, eine Falle usw.)
	\item $\beta$-Diversität: Diversität zwischen den Untersuchungseinheit
	\item $\gamma$-Diversity: Diversität der Gesamtheit aller Untersuchungseinheiten (oft eine Landschaft)
\end{itemize}

\textbf{Partitionierung}
\begin{itemize}
	\item Bezogen auf Artenreichtum: $\alpha \cdot \beta = \gamma$
	\item Bezogen auf $D_{Shannon}$: $\alpha + \beta = \gamma$
	\item Bezogen auf $D_{Gini-Simpson}$: $\alpha + \beta - \alpha \cdot \beta = \gamma$
\end{itemize}

\textbf{Was damit machen?}
\begin{itemize}
	\item Es ist einfach $\alpha$ und $\gamma$ zu berechnen
	\item $\beta$-Diversität wird aus $\alpha$ und $\gamma$ errechnet, wobei $\alpha_{av}$ der Mittelwert über alle Plots ist: $\beta=\gamma-\alpha_{av}$
	\item $\beta$-Diversität wird auch als Maß für „species turnover“ verwendet
\end{itemize}

\newpage
\textbf{$\beta$-Diversität}
\begin{itemize}
	\item ist ein Maß für die Unterschiedlichkeit der Artenausstattung
	\item Wird häufig auch direkt mit multivariaten Vergleichen über Ähnlichkeitsmaße errechnet (Multiple assemblage overlap measures: Morisita-Horn-Index, $C_{qN}$-Index)
\end{itemize}

\textcolor{red}{Ihre Einschätzung: Welche Prozesse befördern $\beta$-Diversität? Wo ist welche Diversitätskomponente wie hoch?}
\begin{itemize}
	\item Moore?
	\item Borealer Wald?
	\item Fynbos (Kapensis)?
	\item Tropischer Regenwald?
	\item Inselarchipele?
\end{itemize}

\newpage
\subsubsection{Funktionelle Diversität}
Kontinuierliche Maße der funktionellen Diversität basieren auf Ähnlichkeit der Arten bzgl. Ihrer Eigenschaften

\textbf{Facetten der Funktionellen Diversität}
\begin{itemize}
	\item \textbf{Funktionelle Identität:} Wo befindet sich die Gemeinschaft im Merkmalsraum?
	\item \textbf{Facetten der funktionellen Diversität}
	\begin{itemize}
		\item \underline{Functional Richness:} Wie groß ist der Merkmalsraum, der von der Gemeinschaft eingenommen wird?
		\item F\underline{unctional Diversity/Divergence/Dispersion:} Wie unterschiedlich sind die Arten im Mittel?
		\item \underline{Functional Eveness:} Wie gleichmäßig sind die Abundanzen der Arten im Merkmalsraum verteilt
		\item \underline{Functional Distinctiveness:} Wie weit ist eine Art im Merkmalsraum von allen anderen entfernt? (Wie „besonders“ ist sie?)
	\end{itemize}
\end{itemize}

\textbf{Funktionelle Identität (FI)}
\begin{itemize}
	\item Mittelwert der Merkmale über alle Arten
	\item Besser: Mittelwert 1 über alle Arten gewichtet mit deren Bedeutung in der Gemeinschaft (Abundanz, Biomasse, Deckung,...)
\end{itemize}

\textbf{Functional richness (FR)}\\
\begin{itemize}
	\item Merkmalsraum, den die Gemeinschaft einnimmt
	\item FR Masszahlen
	\begin{itemize}
		\item Nur ein Merkmal: Spanne zwischen dem kleinsten und dem größten Wert (engl. range)
		\item Zwei und mehrere Merkmale: Fläche des „Convex hull volume (CVH)“: Fläche, die durch eine umhüllende Linie gebildet wird. Die „Eck“-Arten heißen auch Vortex-Arten. Linie darf nicht „nach innen knicken“.
	\end{itemize}
	\item Bemerkung zu FR
	\begin{itemize}
		\item Wird sehr stark durch extreme Arten bestimmt
		\item Für die Vergleichbarkeit wird üblicherweise durch die Gesamtspanne (-fläche, -volumen,...) aller Gemeinschaften geteilt. Dann variieren die Werte'zwischen 0 und 1
	\end{itemize}
\end{itemize}