\section{DS15}

\textbf{Wasserhaushalt}\\
\textcolor{red}{???}
\\\\
\textbf{Regionale Artenpool-Effekte}
\begin{itemize}
	\item Regional unterschiedlichevEvolution, Kontinentaldrift und Verbreitungsbarrieren führend zu unterschiedlichen Artenpools
	\item „Geologisch-Historisches“ Experiment
	\item Große Datenmengen zur Detektion nötig
\end{itemize}

\textbf{Extinktions-Szenatio}
\begin{itemize}
	\item Fallstudie basierend auf Merkmalsmatrix und Abundanzdaten für BCI
	\item Beginnend bei der maximalen Diversität erfolgt das Aussterben der Arten sortiert nach bestimmten Merkmalen
	\item Die Fehlstellen werden Proportional zu den verbleibenden Arten aufgefüllt
	\item Verfolgt wird die relative Änderungen des gewichteten
\end{itemize}

\textbf{Gerichtet versus zufällig}\\
\textcolor{red}{???}
\\\\
\textbf{Fazit}
\begin{itemize}
	\item Eine Vielzahl von Ökosystemprozessen werden durch Artenidentitätseffekte beeinflusst
	\item Diversitätsexperimente zeigen einen großen Einfluss von funktionellen Gruppen. Die meisten Designs lassen es nicht zu, Artidentitätseffekte zu untersuchen.
	\item Invasionsstudien spiegeln nur einen Teil der möglichen Antworten wieder, da sie eine erfolgreiche Invasion voraussetzen (konkurrenzstarke Arten) und daher nicht generalisierbar sind.
	\item Artenpool-Effekte sind viel schwerer zu detektieren. Benötigt sehr viele Beobachtungsdaten (z. B. Inventuren). Aber attraktives Forschungsfeld, das Biogeographie, Diversität und Ökosystemwissenschaften verbindet.
\end{itemize}