\section{DS2}
\subsection{Facetten der Biodiversität}
\begin{itemize}
	\item Molekulare Vielfalt, z. B. Variation zwischen Proteinen (Isoenzyme)
	\item Chemische Vielfalt: z. B. Vielfalt der sekundären Inhaltsstoffe
	\item Genetische Vielfalt: z. B. Genotypen innerhalb einer Art
	\item Phylogenetische Vielfalt: Repräsentanz des „tree of life“
	\item Artenvielfalt: Anzahl und relative Abundanz von Arten
	\item Funktionelle Vielfalt: z. B. physiologische, anatomische, morphologische, demographische, ethologische Vielfalt
	\item Interaktionsvielfalt: z. B. Vielfalt der trophischen Beziehungen sowie aller Sym-, Pro- oder Antibiosen
	\item Ökosystemvielfalt: z. B. Vielfalt der Ökosysteme und Ökosystemprozesse in der Landschaft
\end{itemize}

\subsection{Entwicklung der Biodiversität}
Diversifizerungsmechanismen v.a. Meso-/Känozoische Radiation:
\begin{itemize}
	\item Nach Landgang in Silur zunehmende Nährstoffeinträge vom Land durch organische Partikel
	\item Auseinander brechen von Pangäa erhöht Klimagradienten, Nischenraum und schafft Verbreitungshindernisse, die die Entstehung von Endemismen begünstigen
	\item Zunehmen ausdifferenzierte Baupläne ermöglichen immer größere Spezialisierung und Ausnutzen ökologischer Nischen
\end{itemize}

\textbf{Differentielle Entwicklung in Großtaxa:} Die jeweils neu entwickelten Taxa machen rasch die größte Diversität aus\\

Pionier der Diversitätsforschung: Alexander von Humbolt beschreibt großräumige Diversitätsgradienten\\

erste globale Diversitätskarte: pflanzlichen Diversität nach Wulff (1935), aktualisiert von Mutke \& Barthlott (2005)