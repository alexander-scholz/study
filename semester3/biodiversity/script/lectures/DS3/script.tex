\section{DS3}
\textbf{Fazit aus Montoya et al.:}
\begin{itemize}
	\item Modelle, die neben dem aktuellen Klima auch die Zeit seit der Vereisung beinhalten, erklären die rezenten Diversitätsmuster besser
	\item Unterstützt die Museumstheorie
	\item Problem:
	\begin{itemize}
		\item Differenziert nicht zwischen Artneubildung und Einwanderung
		\item Erklärt nicht den Breitengradienten der Diversität (nur Glazialgeschichte)
	\end{itemize}
\end{itemize}

\textbf{Zeit X Fläche: Ein Test mit Bäumen}
\begin{itemize}
	\item Grundannahme: Zeit und Fläche haben jeweils eigene Erklärungkraft
	\item Problem:
	\begin{itemize}
		\item Evolution über viele Millionen Jahre
		\item Heutige Flächenverteilung der Biome nicht repräsentativ
	\end{itemize}
	\item Lösung: Errechnen des Integrals der verfügbaren Fläche über die geologische Zeit als Prädiktor für Artenreichtum
\end{itemize}

\textbf{Wie hoch ist die Baumdiversität wo?}\\
\begin{tabularx}{0.5\textwidth}{p{0.3\textwidth}|p{0.2\textwidth}}
Biom & \# Baumarten\\
\hline
north american boreal & 61 \\
eurasien boreal & 100 \\
north am. eastern temp. & 300 \\
north am. western temp & 115 \\
europ. temp. & 124 \\
east asien temp. & 729 \\
south am. temp. & 84 \\
australien temp. & 310 \\
neotropics & 22500 \\
asian tropics & 14000 \\
african tropics & 6500 \\
\end{tabularx}
\\\\
\textbf{Korrelationsanalyse}
\begin{itemize}
	\item Keine Korrelation zwischen Artenreichtum und rezenter Biomverteilung!
	\item Bei vier der fünf Biomrekonstruktionen ergeben sich signifikante Effekte mit einer erklärten Varianz von bis zu 60\%.
	\item Legt nahe, dass Zeit und Fläche beide wichtig sind
	\item Sagt aber wenig über die eigentlichen Mechanismen aus (Speziation, Extinktion?)
\end{itemize}

\textbf{Zwischenstand: Zeit / Fläche}
\begin{itemize}
	\item Es gibt Hinweise dafür, dass die verfügbare Zeit „ungestörter Evolution“ (nicht unterbrochen durch Massenextinktionen) positiv mit der Diversität korrelliert.
	\begin{itemize}
		\item Diese Effekte sind besonders stark ausgeprägt für die jüngere Erdgeschichte (Eiszeiten)
		\item Aber es gab auch vorher schon deutliche Breitengradienten der Diversität!
	\end{itemize}
	\item Die Ergebnisse schließen das gleichzeitige Vorhandensein der Effekte von Unterschieden in den Netto-Speziationsraten nicht aus
\end{itemize}

\textbf{Speziationsraten:} Sind die Netto-Speziationsraten in den Tropen höher? Diverse Hypothesen:
\begin{itemize}
	\item Genetic Drift: Kleine Population $\rightarrow$ Genetische
Drift $\uparrow \rightarrow$ Artbildung $\uparrow$ \textcolor{red}{Zirkulär / schwer zu testen}
	\item Klimavariabilität: Milankovich-Zyklen in Tropen $\downarrow \rightarrow$ Vagilität $\uparrow \rightarrow$ Artbildung $\downarrow$ \textcolor{red}{Kaum Daten}
	\item Sympatrische Artbildung $\uparrow$ \textcolor{red}{Kaum Daten}
	\item Metabolismus: Temperatur $\uparrow \rightarrow$ Metabolismus $\uparrow \rightarrow$ Mutation $\uparrow$ \textcolor{green}{Siehe später}
	\item Fläche: Fläche $\uparrow \rightarrow$ Wahrscheinlichkeit der reproduktiven Isolation $\uparrow$ \textcolor{green}{Siehe z.B. Fine \& Rees (oben)}
	\item Toleranzhypothese: Toleranz in den Tropen $\downarrow \rightarrow$ Wahrscheinlichkeit der repr. Isolation $\uparrow$ \textcolor{green}{Plausibel, aber wenig Daten}
	\item Biotische Interaktion: biotische Nischen $\uparrow \rightarrow$ ungerichtete Selektion $\uparrow \rightarrow$ Wahrscheinlichkeit der Divergenz $\uparrow$ \textcolor{green}{Siehe unten}
\end{itemize}

\textbf{Wie alt sind Taxa?} Frage: Seit wann haben sich zwei nächstverwandte Vogel- und Säugetierarten getrennt $\rightarrow$ ergibt eine Altersverteilung; \textbf{Arten höherer Breiten haben sich später getrennt!}
\\\\
\textbf{Fazit: Speziationshypothese}
\begin{itemize}
	\item es gibt Hinweise darauf, dass die molekulare Uhr bei höherer Temperatur „schneller tickt“ (höhere Substitutionsraten).
	\item Diverse Probleme:
	\begin{itemize}
		\item Das erklärt nicht, warum der Breitengradient der Diversität auch für Homoiotherme (Gleichwarm) gilt
		\item Es ist noch unklar, inwieweit die Substitutionsraten ein guter Indikator für Artbildung sind
	\end{itemize}
\end{itemize}

\subsubsection{Biotische Interaktionen}
\begin{itemize}
	\item Temperate Zone: Abiotischer Selektionsdruck (z.B. Spätfrost) ist omnipresent und führt zu gleichgerichteten Anpassungen (targeted Evolution)
	\item In milden Klimaten überwiegt biotischer Selektionsdruck. Dieser ist kleinräumig variabel und unvorhersehbar. Daher sind die Selektionsdrücke entsprechend divers $\rightarrow$ schnellere Divergenz (Evolution with moving target)
\end{itemize}

\textbf{Short-cut: Biotische Interaktion}
\begin{itemize}
	\item Es gibt einige Hinweise auf stärkere Interaktionen in den Tropen (aber auch Gegenbeispiele)
	\item Es gibt Hinweise darauf, dass sich Merkmale, die biotische Interaktion widerspiegeln, schneller evoluieren (z. B. Bestäubungsmodi)
\end{itemize}

\subsubsection{Toleranz-Hypothese}
\begin{itemize}
	\item Tropische Organismen besitzen engere Klimanischen : Höhenzüge wirken daher eher als Barrieren
	\item Die Folge: Schnellere geographische und damit reproduktive Isolierung $\rightarrow$ Divergenz
	\item Der Artenreichtum wird durch die Anzahl von Arten limitiert, die die Umweltbedingungen tolerieren können
	\item Die Umweltbedingungen werden mit der Breite ungünstiger
	\item Zwei Fälle:
	\begin{itemize}
		\item Extremfall: Die Artbildungsrate ist überall gleich (bzw. die Verbreitung ist prinzipiell unlimitiert). Dann ergibt sich die Diversität rein aus der Toleranz (bzw. der differentiellen Extinktion)
		\item Wenn Arten in tropischen Gebieten entstanden sind (oder übrig geblieben sind), müssten sie für eine Ausbreitung polwärts erst Toleranzen entwickeln. Dieser Prozess dauert lange Zeit.
	\end{itemize}
\end{itemize}

\subsubsection{Nischenkonservatismus}
\begin{itemize}
	\item Wenn die Artbildungsrate in den Tropen höher und die Extinktionsraten niedriger sind, warum verbreiten sich die tropischen Arten dann nicht nach Norden aus?
	\item Nischenkonservatismus: Die Anpassungen, die ein Vordringen in kältere Regionen erlauben, sind komplex und werden „selten“ erfunden.
\end{itemize}

\subsubsection{Out of the tropics (OTT)}
\begin{itemize}
	\item Diese Theorie bildet einen Kompromiss. Der Breitengradient hat mehrere Ursachen:
	\begin{itemize}
		\item Höhere Speziationsraten in den Tropen
		\item Geringere Extinktionsraten in Tropen
	\end{itemize}
	\item Gleichzeitig wird davon ausgegangen, dass die hohe Diversität auch in die Extratropen „überschwappt“ (Immigration in den Extratropen hoch).
\end{itemize}

\fcolorbox{red}{white}{\parbox{\linewidth}{\textbf{Fazit}
\begin{itemize}
	\item Die evolutionsbasierten Theorien schließen sich nicht gegenseitig aus!
	\item Sie sind allesamt wahrscheinlicher als die energiebasierten Theorien
	\item Die Evolutionshistorie spielt allgemein ein große Rolle.
	\item Eine synthetische Theorie wie die OTT ist erfolgversprechend.
\end{itemize}
}}